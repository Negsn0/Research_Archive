\documentclass[dvipdfmx]{standalone}
\usepackage{tikz}
\usepackage{pgfplots}
\pgfplotsset{compat=1.18}
\begin{document}
\begin{tikzpicture}

% 縦軸(エネルギーのスケール)
\draw (-0.5, -5.2) -- (-0.5, 0.5) node[above] {MeV};
\foreach \y/\text in {0.0/{0.0}, -10.0/{-10.0}, -20.0/{-20.0}, -30.0/{-30.0}, -40.0/{-40.0}, -50.0/{-50.0}} {
    \draw (-0.5, \y/10) -- (-0.3, \y/10) node[left] {\text\ \ \ };
}

% HOレベル(左側)
\node[left] at (1.25, -5.2) {HO};
\foreach \n/\y/\text in {
    4/-8.757/{$1g, 2d, 3s$},
    3/-17.591/{$1f, 2p$},
    2/-26.424/{$1d, 2s$},
    1/-35.257/{$1p$},
    0/-44.09/{$1s$}} {
    \node[anchor=east] at (0.5, \y/10+0.2) {\scriptsize{$N=\n$}};
    \node[anchor=west] at (1.0, \y/10+0.2) {\scriptsize\text};
    %横線の位置(x,y)
    \draw (0.5, \y/10) -- (1.5, \y/10);
}

% WSレベル(右側)
\node[right] at (3.25, -5.2) {WS};
\foreach \name/\y/\label in {
    1g/-12.713/58,
    2p/-18.782/40,
    1f/-21.829/34,
    2s/-27.754/20,
    1d/-30.084/18,
    1p/-37.636/8,
    1s/-44.09/2} {
    \node[anchor=west] at (4.2, \y/10) {\label};
    \draw (3, \y/10) -- (4, \y/10);
}

% 矢印(HOからWSへの接続)
\if0
\foreach \y in {0.0, -10.0, -20.0, -30.0, -40.0, -50.0, -60.0} {
    \draw[->] (0.5, \y/10) -- (4, \y/10);
}
\fi

\draw[ultra thin]



\node at (2.0,0.0) {$A=100$};
\end{tikzpicture}
\end{document}
