\documentclass[a4paper]{jsreport}
\usepackage{bm}
\usepackage[dvipdfmx]{graphicx}
\usepackage{amsmath,amsfonts,amssymb}
\usepackage{cite}
\usepackage{physics}
\usepackage[version=4]{mhchem}
\usepackage{comment}
\usepackage{float}
\usepackage{url}
\usepackage{booktabs}
\newcommand{\diff}{\mathrm{d}}
\newcommand{\BCSket}{\ket{\text{BCS}}}
\newcommand{\BCSbra}{\bra{\text{BCS}}}
\newcommand{\BCSexp}[1]{\bra{\text{BCS}} {#1} \ket{\text{BCS}}}

\begin{document}
  \chapter{基礎理論の準備}
  ここでは、本研究で必要な知識の概要を掴むことを目的として大まかな「量子多体論・BCS理論」の入門的な知識を示す。
  勿論習熟している読者は読み飛ばしてもらって構わない。
  \section{原子核と量子多体系}
    原子核の構成粒子は、よく知られている通り陽子(proton)と中性子(neutron)の2種類である。  
    この2つの粒子は、スピンが等しく、質量もほぼ同じであり、主に電荷の違いによって区別される。  
    このため、現在ではこれらを同一の粒子の異なる状態と見なし、「核子(nucleon)」と総称するのが一般的である。

    したがって、現代の原子核物理においては、原子核とは有限個の核子($Z$個の陽子と$N$個の中性子)から構成される量子多体系であると認識されている。

    この量子多体系において、全ての核子の状態を厳密に求めることができれば、原子核の性質を完全に記述することができる。しかし、実際には核子間に強い相互作用が存在し、これを全て正確に取り込もうとすると、$Z+N$体問題として扱わなければならず、現実的に解くことは困難である。

    古典力学においてすら、三体問題が一般解を持たないことで知られている。この事実は、量子多体系がいかに難しい問題であるかを直感的に理解する助けとなるだろう。

    このような状況を打開するために導入されるのが、平均場(mean field)近似である。これは、「ある核子が他のすべての核子から受ける相互作用の平均的な効果」を、1体ポテンシャル$V$として取り扱う手法である。

    この平均場の考え方に基づいて構築されたモデルは、特に「マジックナンバー」のような原子核の安定性をうまく再現できることが示されている。これにより、平均場近似が原子核物理において有効な枠組みであることが裏付けられている。

    本研究でも、原子核中の一粒子エネルギー準位を求めるにあたって、この平均場に基づく1体ハミルトニアンを採用する。
    % 原子核の構成粒子は、よく知られている通り陽子(proton)と中性子(neutron)の2種類である。こ
    % の2種類の粒子はスピンが等しく質量がほとんど同じであり、電荷が異なる。
    % このことから、現在ではこの2種類の粒子は同一粒子の異なった状態であると考えて、核子(nucleon)と呼ばれている。
    % つまり、今現在の原子核は、有限個の核子($Z$個の陽子と$N$個の中性子)から構成されている量子多体系と認識されている。
    % この量子多体系において、すべての核子についての状態を厳密に求めることで原子核の状態が得られるが、
    % 多体系の1粒子Hamiltonian中にはすべての粒子間での相互作用が含まれるため、単純に($Z+N$)体問題となる。
    % 有名な話として、古典力学において三体問題が厳密に解く(一般解を求めること)ができないというものがあり、
    % このことからも相互作用を踏まえて厳密に状態を求めることが不可能であるというのは理解できると思う。
    % ではどのように考えていくかというと、$Z+N$個の粒子それぞれが作る場を、平均場としてポテンシャルで採用する。
    % この考えは妥当であり、原子核のMagic numberをうまく説明することができた。
    % 従って原子核中の一体粒子エネルギーを求める際にはこの方法を用いる。

  \section{ペアリング相関}
    原子核において、以下のような実験的事実が知られている:

  \begin{itemize}
    \item[(i)] 中性子数および陽子数がともに偶数の核(以下、偶-偶核)では、基底状態の全角運動量$I$およびパリティ$\pi$が例外なく $I^\pi = 0^+$ である。
    \item[(ii)] ほとんどの奇数質量数の核においては、基底状態のスピンとパリティが、最後の奇数番目の粒子が占める単一粒子状態 $a \equiv \{n_a, \ell_a, j_a; \text{荷電}~q\}$ のそれと一致しており、$I = j_a$, $\pi = (-1)^{\ell_a}$ となる。
  \end{itemize}

  これらの事実は、同種粒子(中性子–中性子、または陽子–陽子)同士が対を作り、特に $J^\pi = 0^+$ の状態にあるときに、通常よりも強い結合エネルギーを得ていることを示唆している。

  すなわち、偶数核ではすべての粒子がペアを形成し安定する一方で、奇数核では1つの粒子がペアを持たないため、その存在が基底状態の量子数に反映される。

  この「対を作ることで安定化する」という性質は、実験的に観測されるペアリングギャップ(例えば、中性子を1つ取り除いたときに現れるエネルギー差)として定量的にも確認されている。

  また、このようなペア形成は、固体物理におけるBCS理論におけるCooper対と類似しており、原子核にBCS理論を応用する理論的根拠ともなっている。

\end{document}